

\section{Procesos a tasa constante}

En la Naturaleza existe una gran cantidad de procesos que a lo largo del tiempo cambian o evolucionan a {\it tasa constante}. En esta sección haremos las definiciones esenciales y en las próximas veremos ejemplos de ellos.

Sin embargo, primero queremos precisar que cuando hablemos de {\it Naturaleza } entenderemos que comprende al mundo vivo --humano o no humano-- y los fenómenos que involucran a la materia inerte.

Vamos a  comenzar por lo básico. Supongamos que tenemos una magnitud $p(t)$ que es función del tiempo. Dicha magnitud puede representar a cualquier variable de interés de estudio: tamaño de una población, cantidad de dinero, número de enfermos, cantidad de material radioactivo, etcétera. Empezaremos por suponer que $p(t)$ solamente puede evaluarse en valores discretos del tiempo por lo cual es de interés estudiar la sucesión:
\[
\{p(t)\}_{t=0}^n = p(0),p(1),p(2)\hdots p(n)
\]
\noindent donde $n$ es un valor arbitrario del tiempo.

$p(n+1)-p(n)$ representa el incremento de la variable $p$ en una unidad de tiempo, mientras que:

 
\begin{equation} \label{eq:1}
 \dfrac {p(n+1) -p(n) }{p(n)}
\end{equation}

 
 Es el incremento de la magnitud $p$ con respecto a los existentes. Dado que el valor de un cociente no se altera si se multiplica por el mismo número su numerador y su denominador, se puede elegir un factor adecuado de manera que el denominador de la expresión \ref{eq:1} sea $100$ en cuyo caso diremos que se tiene  \emph{el incremento porcentual} o bien el \emph{incremento per cápita} de la magnitud $p(t)$ por unidad de tiempo. Si suponemos que el incremento porcentual es una constante $q$ es decir:
 
\begin{equation} \label{eq:2}
 \dfrac {p(n+1)-p(n)}{p(n)}=q
\end{equation}

\noindent Entonces el incremento por unidad de tiempo se puede expresar como: 

 \begin{equation} \label{eq:3}
 p(n+1) -p(n)=p(n)q
\end{equation}

De donde se deriva el hecho de que para medir el incremento de la magnitud $p$ en una unidad de tiempo, da lo mismo restar el valor futuro al actual que multiplicar el valor actual por la tasa de crecimiento porcentual por unidad de tiempo. Un despeje más nos lleva a:

 \begin{equation} \label{eq:4}
 p(n+1)=p(n)+p(n)q
\end{equation}

\noindent y arribamos a una verdad de perogrullo: el valor futuro de la magnitud $p(t)$ es igual a lo que se tenía más lo que aumentó. Más aún:

 \begin{equation} \label{eq:5}
 p(n+1)=p(n)(1+q)
\end{equation}

Entonces, el valor futuro es el valor anterior multiplicado por el factor $1+q$ que por esta razón recibe le nombre de \emph{tasa de reemplazo}.

Si ponemos $n=0$ y con la igualdad \ref{eq:5} calculamos $p(1)$ que, a su vez, nos sirve para calcular $p(2)$ y así sucesivamente. Llegamos la la expresión:

 \begin{equation} \label{eq:6}
 p(n)=p(0)(1+q)^n
\end{equation}

Esta es una \emph{fórmula predictiva}. Si se conoce la condición inicial $p(0)$ y la tasa de crecimiento \emph{per cápita} por unidad de tiempo, entonces la expresión \ref{eq:6} nos da la capacidad de conocer el valor futuro de $p$ para cualquier tiempo transcurrido mientras el tiempo se mida en unidades discretas.

\begin{ejemplo}
		
		
		\noindent La bacteria \emph{Escherichia coli} tiene una forma aproximadamente cilíndrica en la que la altura del cilindro mide cerca de 1$\mu$. Se reproduce por bipartición cada hora. La pregunta es ¿Sí se tiene inicialmente una bacteria, cuantas habrá después de una semana?.
		
		Si $p(0)=1$ y la unidad de tiempo es una hora, entonces $p(1)=2$ y, por lo tanto 
		 
\begin{equation} \label{eq:7}
 q=\dfrac{p(1)-p(0)}{p(0)}=\dfrac{2-1}{1}=1=\dfrac{100}{100}=100\%
\end{equation}

Una semana tiene 604800 segundos, por lo tanto y de acuerdo con nuestra fórmula predictiva \ref{eq:6}, dado que $q=1$ el resultado del ejercicio es:

\[
p(168)=2^{168}\approx 10^50
\]
Se deja al lector que realice el siguiente cálculo: si la bacteria fuera un cubo con $1\mu$ de lado ¿Qué volumen ocupan las bacterias que resultan de la reproducción a partir de una sola después de una semana?

\end{ejemplo}

\begin{ejemplo}

		\noindent La tasa de crecimiento de la República Mexicana en el 2015 fue 1.4\% anual. En ese mismo año la población total eran 9 millones. ¿Cuántos mexicanos habrá en el año 2050?
		
		En este caso $q=1.4\%=1.4/100=0.014$ y la $p(0)=9$. Donde hemos puesto el cero del tiempo en el 2015. Si ponemos que el tiempo transcurrido es de 35 años, la fórmula predictiva \ref{eq:6} queda como:
		
		\[
		p(35)=9\times(1.014)^{35}=9\times1.627=14.64 \textrm{ millones de habitantes}
		\]
		Esa es nuestra predicción. Sin embargo, la CONAPO pronostica una población de $7.7$ millones ¿Cómo explica usted la discrepancia?
		

\end{ejemplo}

En los dos ejemplos, tenemos sendas tasas de crecimiento positivas. Ahora vamos a ver que también se puede hablar de decrecimiento a tasa constante, veamos un ejemplo ilustrativo:

\begin{ejemplo}

\noindent  La cima del monte Everest se encuentra a $8,848$ metros sobre el nivel del mar y la presión atmosférica es de $0.33$ atmósferas. Se trata de encontrar la presión atmosférica en la Ciudad de México --$2240$ metros sobre el nivel del mar--.

En este caso, la variable independiente no es el tiempo sino la altura a partir del nivel de mar. A la altura cero la presión es $1$atm. 

\begin{align*}
  p(0) &= 1 \\
  p(8,848) &= 0.33p(0)
\end{align*}

Por lo tanto:

\[
q=\dfrac{p(1)-p(0)}{p(0)}=\dfrac{0.33-1}{1}=-0.66\%
\]

Es la tasa de decaimiento de la presión atmosférica por unidad de altura pero nos topamos con una dificultad técnica.La tasa $q$ que hemos calculado está definida cuando la unidad de altura es $1$ \emph{everest}. Desde luego que quisiéramos una tasa que estuviera, digamos, en metros. Vamos a ver que un razonamiento sencillo nos permitirá hacerlo:

\[
p(1)=p(0)(1+q_e)^1
p(1)=0.33p(O)
\]

Donde hemos llamado $q_e$ a la tasa medida en \emph{everests} pero debe de existir una tasa $q_m$ medida en metros que haga el mismo trabajo. Es decir;

\[
p(8,848)=p(0)(1+q_m)^{8,840}=0.33p(0)
\]
Usamos la segunda igualdad:


\begin{align*}
p(8,848) &=p(0)0.33\\
p(8,848) &=p(0)(1+q_m)^{8,848}
\end{align*}

Igualando los segundos términos y despejando la $q_m$, finalmente llegamos a:

\[
q_m=\sqrt[8848]{0.33}-1=0.0001259
\]



 Entonces, para saber la presión atmosférica en la Ciudad de México simplemente hay que calcular:

\[
p(2400)=p(0)(1+q_m)^{2400}=p(0)(1-0.0001259)^{2400}=p(0)0.76
\]

Y llegamos a la respuesta correcta: La presión atmosférica en la Ciudad de México es el $76\%$ de la presión al nivel del mar.

\end{ejemplo}



En el primer ejemplo nos daban como datos la población inicial y la población una unidad de tiempo después. En el segundo, se nos proporcionó la población inicial y la tasa de crecimiento. En el tercero aprendimos como calcular tasas de \emph{crecimiento equivalentes}. Esto va a resultar muy útil cuando queramos cambiar una tasa expresada en alguna unidad a la equivalente pero dada en submúltiplos de la unidad. Es un error muy común pero ampliamente usado en los medios de comunicación que si se tiene una tasa, digamos, anual, para obtener la tasa mensual se divide entre doce y ya. 

Si suponemos que tenemos un proceso que crece a la tasa de $q=120\%$ anual y nos dejamos guiar por los que leen las noticias, responderemos que la tasa mensual es del $10\%$

Llamemos $q_m$ a la tasa mensual equivalente y procedamos según lo ilustrado en el ejemplo número $3$: 

\begin{align*}
p(1)=p(0)(1+q)=p(0)(2.2)\\
p(12)=p(0)(1+q_m)^{12}=p(0)(2.2)
\end{align*}

Igualando las dos expresiones y siguiendo el procedimiento del ejemplo 3, se tiene que:

\[
q_m=\sqrt[12]{1+q}=\sqrt[12]{2.2}
\]

En nuestro caso $q=1.2$ por lo que después de hacer las cuentas llegamos a:

\[
q_m=0.0679
\]

De manera que la respuesta correcta es que la tasa mensual equivalente a una tasa anual del $120\%$ es del $6.79\%$. Por favor, verifíquelo. Es claro que sería un ejercicio inútil tratar de explicarlo a comentadores de noticias.

En un resumen parcial rápido, podemos decir que hemos aprendido lo que es una tasa de crecimiento per cápita y por unidad de tiempo. Tenemos la fórmula $6$ que es una poderosa herramienta de predicción siempre y cuando se mantenga firme la premisa de que \emph{la tasa sea constante}. También sabemos cómo proceder si conocemos la tasa $q$ para una unidad de tiempo (o más general; para una unidad de la variable independiente) y deseamos conocer la tasa equivalente para una subunidad. 

Si se conocen: i) La tasa de crecimiento $q$, ii) la condición inicial $p(0)$ y las unidades de tiempo $n$ que han de transcurrir, entonces podemos predecir el valor de la magnitud $p(n)$. A este esquema de trabajo, le vamos a llamar el \emph{problema directo}.

Sin embargo, también es un problema de importancia práctica el preguntarse cuántas unidades de tiempo $n$ han de transcurrir para que tengamos un valor prefijado de la magnitud $p(n)$

\begin{ejemplo}

Si la presión atmosférica disminuye con la altura a tasa constante ¿A qué altura se tendrá el $80\%$ de la presión al nivel del mar?
Recordemos la fórmula predictiva \ref{eq:6}

 \begin{equation*} 
 p(n)=p(0)(1+q)^n
\end{equation*}

Ahora la incógnita es la variable independiente $n$ mientras que los datos son la tasa $q_m$ (recordar la notación del ejemplo 3), la condición inicial $p(0)$ y la magnitud $p(n)$. Despejemos $n$ de la expresión anterior:

\[
n=\dfrac{\log\frac{p(n)}{p(0)}}{\log(1+q_m)}
\]

Y dado que sabemos que $q_m=0.01259$ y que $p(n)=0.8p(0)$, haciendo las sustituciones pertinentes\footnote{De aquí adelante utilizaremos la notación $\log()$ para el logaritmo natural. Todos los resultados son aproximados pues hemos truncado las los números a pocas cifras después del punto decimal.} llegamos a la respuesta: $n=1780.5135$. A la altura aproximada de $1780$ metros, la presión atmosférica es el $80\%$ de la que hay al nivel del mar.

\end{ejemplo}

Hasta ahora, todo parece ir sobre ruedas pero el ejemplo anterior amerita una mayor reflexión. El resultado de ls operaciones es $n=1780.5135$ y arbitrariamente nos hemos conformado con la aproximación truncando las cifras no enteras ¿Realmente no lo podemos hacer mejor?. 

Regresemos a la fórmula predictiva \ref{eq:6}: 

\[
p(n)=p(0)(1+q)^n
\]

En los párrafos anteriores insistimos en la variable independiente $n$ tenía que ser un número entero de modo que no podemos apostar a que la respuesta del ejemplo anterior sea rigurosamente la obtenida. Sin embargo, ya hemos desarrollado un método para encontrar la tasa de crecimiento per cápita para submúltiplos de la tasa por unidad de tiempo. Ahora supongamos que existe un número $k$, al que llamaremos la \emph{tasa instantánea} de crecimiento que cumple:

\begin{align}
    p(t)=p(0)(1+q)^t=p(0)e^{kt}
\end{align}

Que sea válida para $t$ entero pero que interpole los valores de $p(t)$ para cualquier valor de $t$ en los números reales. Si se conoce la tasa $q$ por unidad de tiempo, entonces es fácil calcular la tasa instantánea $k$



\section{El crecimiento Malhusiano}








