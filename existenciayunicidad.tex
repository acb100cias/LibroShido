\section{Existencia y unicidad de las soluciones}

\begin{definicion}
Consideremos una función $f:\mathbb{R}^{n+1}\longrightarrow\mathbb{R}^n$ definida en $|t-t_0|\leq a$ con $x\in D\subset\mathbb{R}^n$. Diremos que $f$ satisface la condición de Lipschitz respecto a $x$ si en el conjunto $\left[t_0-a,t_0+a\right]\times D$ se tiene que:
$$
||f(t,x_1)-f(t,x_2) ||\leq L ||x_1-x_2||
$$
donde $x_1$, $x_2\in D$ y $L>0$. $L$ es llamada constante de Lipschitz para $f$.
\end{definicion}

\begin{obs}
Si una función cumple la condición de Lipschitz, entonces es continua y si la función es de clase $C^1$ entonces necesariamente cumple la condición de Lipschitz
\end{obs}

\begin{teorema}
Consideremos el problema de condiciones iniciales
\begin{eqnarray}
\dot{x}&=&f(t,x)\nonumber\\
x(t_0)&=&x_0\label{1}
\end{eqnarray}
con $x\in D\subset\mathbb{R}^n$, $|t-t_0|\leq a$ y $D=\{x | ||x-x_0||\leq d\}$, $a,d>0$. Si la función $f$ satisface que:
\begin{enumerate}
    \item Es Lipschitz continua en $G=[t_0-a,t_0+a]\times D$
    \item Es Lipschitz continua en $x$
\end{enumerate}
entonces el problema \ref{1} tiene solución única en $|t-t_0|\leq \inf (a,\frac{d}{M})$ con $M=sup_{G} ||f||$

\end{teorema}

\begin{teorema}{\bf Desigualdad de Gronwall}
Supongamos que para $t\in(t_0,T_0+a)$, $a>0$ se cumple que:
$$
\varphi(t)\leq \delta_1\int_{t_0}^{t}\psi(s)\varphi(s)\mathrm{d}s +\delta_2
$$
con $\varphi\leq 0$ y $\psi\leq 0$ continuas sobre $(t_0,T_0+a)$, $\delta_i >0$ . Entonces para $t\in (t_0,T_0+a)$:

$$
\varphi(t)\leq \delta_2e^{\delta_1\int_{t_0}^{t}\psi(s)\varphi(s)\mathrm{d}s}
$$

\end{teorema}

\begin{proof}
Como $\varphi(t)\leq \delta_1\int_{t_0}^{t}\psi(s)\varphi(s)\mathrm{d}s +\delta_2$ , tenemos que:
$$
\frac{\varphi(t)}{\delta_1\int_{t_0}^{t}\psi(s)\varphi(s)\mathrm{d}s +\delta_2}\leq 1
$$
multiplicando por $\delta_1\psi(t)$ e integrando de $t_0$ a $t$:
$$
\int_{t_0}^{t}\frac{\delta_1\psi(s)\varphi(s)\mathrm{d}s}{\delta_1\int_{t_0}^{t}\psi(\tau)\varphi(\tau)\mathrm{d}\tau +\delta_2}\leq \delta_1\int_{t_0}^{t}\psi(s)\mathrm{d}s
$$
así
$$
ln(\delta_1\int_{t_0}^t\psi(s)\varphi(s)\mathrm{d}s+\delta_2)-ln\delta_2\leq\delta_1\int_{t_0}^t\psi(s)\mathrm{d}s
$$
y por tanto
$$
\varphi(t_0\leq\delta_1\int_{t_0}^t\psi(s)\varphi(s)\mathrm{d}s+\delta_2\leq\delta_2e^{(\delta_1\int_{t_0}^t\psi(s)\mathrm{d}s)}
$$
\end{proof}
