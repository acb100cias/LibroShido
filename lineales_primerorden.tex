\section{Ecuaciones Lineales de Primer Orden}

Una ecuación diferencial lineal de primer orden tiene la forma

$$
\frac{\mathrm{d}y}{\mathrm{d}x}+p(x)y=f(x)
$$
donde las funciones $p$ y $f$ son continuas en la región donde existe la solución. Si $f(x)\equiv0$ entonces llamaremos a la ecuación homogénea.

En este caso la ecuación lineal puede reescribirse como
$$
\frac{\mathrm{d}y}{\mathrm{d}x}=-p(x)y
$$
la cual claramente es de variables separables. En efecto obtenemos la ecuación integral equivalente 
$$
\int\frac{\mathrm{d}y}{y}=-\int p(x)\mathrm{d}x
$$
integrando obtenemos
$$
\ln |y|=-\int p(x)\mathrm{d}x+\ln c_1,\qquad c_1>0
$$
y por tanto
$$
y=ce^{-\int p(x)\mathrm{d}x}, \quad c\neq 0
$$

Observese que aldividir entre $y$ perdimos la solución trivial $y\equiv 0$, pero podemos recuperarla tomando $c=0$


