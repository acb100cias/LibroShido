\section{Ecuaciones Lineales de Primer Orden}

Una ecuación diferencial lineal de primer orden tiene la forma

$$
\frac{\mathrm{d}y}{\mathrm{d}x}+p(x)y=f(x)
$$
donde las funciones $p$ y $f$ son continuas en la región donde existe la solución. Si $f(x)\equiv0$ entonces llamaremos a la ecuación homogénea.

En este caso la ecuación lineal puede reescribirse como
$$
\frac{\mathrm{d}y}{\mathrm{d}x}=-p(x)y
$$
la cual claramente es de variables separables. En efecto obtenemos la ecuación integral equivalente 
$$
\int\frac{\mathrm{d}y}{y}=-\int p(x)\mathrm{d}x
$$
integrando obtenemos
$$
\ln |y|=-\int p(x)\mathrm{d}x+\ln c_1,\qquad c_1>0
$$
y por tanto
$$
y=ce^{-\int p(x)\mathrm{d}x}, \quad c\neq 0
$$

Observese que al dividir entre $y$ perdimos la solución trivial $y\equiv 0$, pero podemos recuperarla tomando $c=0$.

Para el caso no-homogéneo aplicaremos el llamado metodo de varición de constantes. Comenzamos integrando la ecuación homogenea asociada.
$$
\frac{\mathrm{d}y}{\mathrm{d}x}+p(x)y=0
$$
obteniendo la solución general
$$
y(x)=ce^{-\int p(x)\mathrm{d}x}
$$
cuando $c$ es una constante $y(x)$ es una solución para la homogenea. 

Ahora supongamos que $c$ es una función desconocida que depende de $x$. Derivando $y(x)$ tenemos:
$$
\frac{\mathrm{d}y}{\mathrm{d}x}=\frac{\mathrm{d}c}{\mathrm{d}x}e^{-\int p(x)\mathrm{d}x}-c(x)p(x)e^{-\int p(x)\mathrm{d}x}
$$
si esta es una solución para la ecuación no-homogénea entonces debe satisfacer
$$
\frac{\mathrm{d}c}{\mathrm{d}x}e^{-\int p(x)\mathrm{d}x}-c(x)p(x)e^{-\int p(x)\mathrm{d}x}+p(x)c(x)e^{-\int p(x)\mathrm{d}x}=f(x)
$$
así que
$$
\frac{\mathrm{d}c}{\mathrm{d}x}=f(x)e^{\int p(x)\mathrm{d}x}
$$
integrando obtenemos
$$
c(x)=\int f(x)e^{\int p(x)\mathrm{d}x}\mathrm{d}x+c_1
$$
y por tanto la solución de nuestra ecuación no-homogenea tiene la forma
$$
y(x)=c_1e^{-\int p(x)\mathrm{d}x}+e^{-\int p(x)\mathrm{d}x}\int f(x)e^{\int p(x)\mathrm{d}x}\mathrm{d}x
$$

Un aspecto importante del proceso anterior es que la solución general de la ecuacion lineal no-homogéneaestá dada como la suma de la solución de la ecuación homogénea asociada y una solución particular de la ecuación no-homogénea, que corresponde a tomar el valor $c_1=0$
$$
e^{-\int p(x)\mathrm{d}x}\int f(x)e^{\int p(x)\mathrm{d}x}\mathrm{d}x
$$

\begin{ejemplo}
 Consideremos la ecuación
 $$
 \frac{\mathrm{d}y}{\mathrm{d}x}-\frac{y}{x}=x^2
 $$
 apliquemos el método de variacion de constantes
 
\end{ejemplo}

